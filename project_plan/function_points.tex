\section{Function Points}
The Functional Point approach is an algorithmic technique, defined in 1975 by Allan Albrecht, that allows to evaluate the effort needed to develop a project. The functionalities list has been obtained from the RASD and the Design Document evaluating every specific function.
The functionalities have been groped in:
\begin{itemize}
	\item \textbf{Internal Logic Files:} homogeneous set of data used and managed by the application;
	\item \textbf{External Interface Files:} homogeneous set of data used by the application but generated and maintained by other applications;
	\item \textbf{External Input:} elementary operation to elaborate data coming from the external environment;
	\item \textbf{External Output:} elementary operation that generates data for the external environment\footnote{It usually includes the elaboration of data from logic files};
	\item \textbf{External Inquiry:} elementary operation that involves input and output\footnote{Without significant elaboration of data from logic files}.
\end{itemize}
The following table specifies the function points values we have used:

\begin{tabular}{|c|c|c|c|}\hline
	\multirow{2}{*}{Function Type} & \multicolumn{3}{c|}{Complexity} \\ \cline{2-4}
	 & Simple & Medium & Complex \\ \hline
	Internal Logic File	& 7 & 10 & 15 \\ \hline
	External Interface File	& 5 & 7 & 10 \\ \hline
	External Input	& 3 & 4 & 6 \\ \hline
	External Output	& 4 & 5 & 7 \\ \hline
	External Inquiry& 3 & 4 & 6 \\ \hline
\end{tabular} 

Based on the specified combination of program characteristics, a weight is associated with each of these FP counts; the total is computed by multiplying each raw count by the weight and summing all partial values.
$$\sum \#ElementsOfGivenType * Weight$$
\subsection{Internal Logic Files}
In the system, there are different kind of ILFs. For the booking history, bookings info and taxi list we have identified a simple weight complexity; the queue management, instead, has a complex weight since has to manage a high number of taxis, zones and queues.
$$FP_{ILF}=7*3+15*1=36$$
\subsection{External Interface Files}
In the system, there are only one EIF which is the location management. It is a simple complex weight since it only has to retrieve information from the taxis GPSs. We can consider this as a simple complexity.$$FP_{EIF}=5*1=5$$
\subsection{External Input}
There are several external inputs in the system. On the side of the user, there are the bookings registration, long-term reservations modification and long-term reservations deletion. On the side of the taxi, there are the login/logout and the registration. All the inputs can be considered as simple since they do not require much data manipulation. $$FP_{EInput}=3*5=15$$
\subsection{External Output}
The only external output in the system are the different types of notification. We consider them as medium complexity since there are different types of notification. $$FP_{EO}=5*1=5$$
\subsection{External Inquiry}
There are several external inquiries in the system. On the user side, there are the company information and the long-term reservations information. On the taxi side, there are the rides information and the current ride information. We consider the user long-term reservations information and the taxi ride history as medium complexity; instead, we consider the company information and the current ride information as simple. $$FP_{EInquiry}=3*1+4*2=11$$

\subsection{Summing Up}
Summing all the partial results of the previous sections, we obtain the Unadjusted Function Point.
$$UFP=FP_{ILF}+FP_{EIF}+FP_{EInput}+FP_{EO}+FP_{EInquiry}$$ $$UFP=36+5+15+5+11=72$$