\section{Overall description}
	\subsection{Product perspective}
		\emph{Whereas the system is too much little, considering its dimensions, we have consolidate the six subsections of the IEEE Std 830-1998}.
		\vspace{10pt}
		
		In the case of web application, a user must have a computer to access to the service site. It's necessary also a browser to allow navigation. This application isn't a weak one and, so, it doesn't imply a heavy work of the primary memory. 
		Through the site, the user can update himself/herself about this service and, then he/she can book a taxi, for a short-term reservation or a long-term one. After that, he/she can modify his/her reservation or delete it. In addition of these actions, the user can view his/her reservation and he/she can view the system communications.
		
		In the case of mobile application, a user must have a smart phone with an Internet connection to access to the service app. It's necessary that the cell phone has got sufficient memory to allow the installation of this app, that is compatible with all of the SOs (i.e. Android, iOS and Windows Phone).
		Through this user-friendly mobile application, the user can do the same stuffs, using the web application: in particular, he/she can book a taxi (always for a short-term or a long-term reservation), he/she can modify his/her reservation and he/she can delete it.
		Instead, the taxi driver, through the mobile application, can view: some information about the current ride; the notifications from the system; previous rides. Then the taxi driver can send to the system notifications about delays.  
	\subsection{Product functions}
	\subsection{Constraints}
	\subsection{Assumptions and dependencies}