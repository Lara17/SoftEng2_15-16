\section{Integration Strategy}
\subsection{Entry Criteria}
	Before starting the test phase, the following documents must be ready and read:
	\begin{itemize}
		\item The Design Document of previous delivery to understand how the software is built, in particular, the component view which specifies how the various modules of the system are connected to each other.
		\item Sections 2.3 and 2.4 of this document that explains which test strategy we use and the testing order of every module and submodule.
		\item How the database is built and which data it contains.
		
		Also before testing each module that comes after some other modules in the testing order:
		
		\item Test result of previous order and the drivers required to run that test.
	\end{itemize} 
\subsection{Elements to be Integrated}
		\begin{center}
			\includegraphics[width=0.95\textwidth]{./images/test_modules.png}
		\end{center}
		The system to be tested is represented in this image. Every submodule has to be tested after all his predecessors in testing order, specified in section 2.5, have been tested. The system can be divided in 3 modules, and every module can be divided in submodules.
		\subsubsection{Modules to be Tested}
		
		\begin{itemize}
			\item \textbf{Application Server}
			\begin{itemize}
				\item Booking Manager
				\item Taxi Authentication Manager
				\item Queue Manager
				\item Taxi Registration Manager
				\item Location Manager
				\item Delay Notification Manager
			\end{itemize}
			\item \textbf{Taxi Interface}
			\begin{itemize}
				\item Registration Interface
				\item Login Interface
				\item Information Current Ride Interface
				\item Rides History Interface
				\item Notification Delay Interface
			\end{itemize}
			\item \textbf{User Interface}
			\begin{itemize}
				\item Booking Interface
				\item Long-term Reservation Delete Interface
				\item Long-term Reservation Info Interface
				\item Information Interface
				\item Long-term Reservation Modification Interface
			\end{itemize}
		\end{itemize}
\subsection{Integration Testing Strategy}
The strategies we could use are: top-down, bottom-up, sandwich, thread and critical modules. After some considerations, we have understood that, among them, the preferable strategies are the first two ones written. Our choice depends on the three facts below:
	\begin{itemize}
		\item besides the sandwich strategy is flexible and adaptable, it's complicated to plan;
		\item it's not better to use the thread strategy: if we consider portions of several modules, some problems can occur because some modules depend on others, entirely and not only on a part of them.
		\item the critical modules strategy is more imprecised than the others.      
	\end{itemize} 
It's indifferent to use the bottom-up strategy or the top-down one. In this document, we have focused on the first strategy nominated.
\subsection{Sequence of Component/Function Integration}
\subsubsection{Integration Sequence}
Considering the bottom-up strategy, first of all, we will test the module called "Application Server", and, in particular, its submodules, represented in the image below:
    \begin{center}
	    \includegraphics[width=0.21\textwidth]{./images/ModuleApplicationServer.png}~\\[1.5cm] 
    \end{center}	
After that, we can test two modules:
	\begin{itemize}
		\item the one named "Taxi Interface", represented in the image below
		    \begin{center}
			    \includegraphics[width=0.60\textwidth]{./images/ModuleTaxiInterface.png}~\\[1.5cm] 
			\end{center}
		\item the one called "User Interface", corrisponding to the diagram below
		    \begin{center}
			    \includegraphics[width=0.85\textwidth]{./images/ModuleUserInterface.png}~\\[1.5cm] 
			\end{center}
		The arrows represent the testing order.
	\end{itemize}
