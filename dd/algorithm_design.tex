\section{Algorithm Design}
\subsection{Queue manager}
\begin{algorithmic}
	\Function{manageQueue}{zone}
		\State $queue\gets getZoneQueue(zone)$
		\While{queueIsEmpty}
				\State $adjacentZone \gets getAdjacentZone(zone)$
				\State $queue\gets getZoneQueue(adjacentZone)$
		\EndWhile
		\State $taxiCode \gets getTaxi(queue)$
		\State \Return{$taxiCode$}
	\EndFunction
\end{algorithmic}


\vspace{0.5cm}

The above pseudo code describes how the Queue Manager organizes the queue of every zone. 
First of all, the function "manageQueue" will save in the variable "queue" the queue, which will be the return of the function "getZoneQueue". 
As long as the queue, saved in the variable with the same name, is empty, this function will check the queues of the adjacent zones and it will save in the usual variable the first queue not empty.
In this way, the first taxi of the considered queue will be saved in the variable "taxiCode", that will be returned.

\newpage
\subsection{Reservation Manager}
\begin{algorithmic}
	\Function{manageBooking}{origin, time, destination}
		\If {$!originIsValid(origin)$ {\bf or} $!timeIsValid(time)$ {\bf or} $(!originIsValid(origin)$ {\bf and} $!timeIsValid(time))$} 
			\State \Return error 
		\EndIf
		\If{$time = CURRENT$\textunderscore  $TIME$}
			\State $booking \gets createShortTermReservation(origin, destination)$
			\State $zone \gets getZone(origin)$
			\While{!positiveResponse}
				\State $taxi \gets manageQueue(zone)$
				\State $response \gets rideNotification(taxi)$
			\EndWhile	
		\Else 
			\State $booking \gets createLongTermReservation(origin, time, destination)$
			\State $saveBooking(booking, database)$
			\State \Return
		\EndIf
		\State $sendConfirmation(booking)$
		\State \Return
	\EndFunction
\end{algorithmic}

\vspace{0.5cm}

The above pseudo code describes how the Booking Manager organizes the bookings. 
First of all, the function "manageBooking" checks if the origin address and/or the time, inserted by the user, is/are valid or not: in the worst case, there will occur an error. 
Otherwise, if the time is the current one, in the variable "booking", there will be saved a short-term reservation, which will be the return of the function "createShortTermReservation". In the variable "zone", this function will save the zone in which there is the origin address specified by the user. 
In the while cycle, that will occur as long as the taxi response will not be positive, thanks to the functions "manageQueue" and "rideNotification", a taxi will be chosen to do the ride. 

If the time is not the current one, in the variable "booking", there will be saved a long-term reservation, which will be the return of the function "createLongTermReservation".
Then, the information about the booking reserved will be saved in the database, through the function "saveBooking".
Finally, a confirmation of the booking realized correctly will be send to the user.