\section{Introduction}
\subsection{Purpose}
	This document is the Design Document (DD) of a system, called myTaxiService, used to manage  a taxi service in a city. The main goals of this document are to specify how the system has to be build in terms of software and hardware architecture, in particular, specifying how the software architecture is built, what kind of styles are used and which kind of tiered architecture is used for the hardware part.
\subsection{Scope}
	The scope of this system is to manage taxis in a city. A town is divided in zone of 2 square kilometers and, for each zone, the system defines a queue, composed by the identifier, which is the vehicle plate, of free taxis in that specific zone. A user can require a taxi ride from a zone, but he/she can also book one for another moment, using the web application or the mobile one. Furthermore, about long-term reservations, after the user has created one, he/she is able to modify the date or the hour or both of his/her booking, and he/she can delete it. 
\subsection{Definition, acronyms and abbreviations}
	\subsubsection{Definition}
	\subsubsection{Acronyms}
		 \begin{itemize}
		 	\item RASD: Requirements Analysis and Specification Document from previous delivery
		 \end{itemize}
	\subsubsection{Abbreviations}
		\begin{itemize}
			\item Gi: goal number i, that refers to the corresponding goal in the RASD.
		\end{itemize}
\subsection{Reference Documents}
	To redact this document, we used the following other documents as references:
	\begin{itemize}
		\item Requirements Analysis and Specification Document from previous delivery
		\item Design Document Table of Content 
	\end{itemize}
\subsection{Document Structure}
	This document contains the schema that represents the software and the hardware architectures, sequence diagrams that show how the system works and the design choices made to build the system. After that, there are some implementations in pseudo code, user interfaces and all the requirements present in the RASD, mapped in this document.